\documentclass{article}
\usepackage{../header}

\begin{document}

\Makepagesectionhead{MATH 631}{Sheaves}{ARessegetes Stery}

\tableofcontents  
\clearpage

\setcounter{section}{-1}

\section{A Brief History of Algebraic Geometry}

\textstart
The initial purpose of studying algebraic geometry is to study the geometry specified by some algebraic relations, of which the most typical is roots of polynomials. The classical formulation (1900s) is as follows: given variables $x_1, \dots, x_n$ and polynomials $f_1, \dots, f_m \in \C[x_1, \dots, x_n]$, consider the space constitutes of the root of polynomials
\[
    \{ (x_1, \dots, x_n) \in \C^n \mid f_1(x_1, \dots, x_n) = \cdots = f_m(x_1, \dots, x_n) = 0 \}
\]
This is often referred to as \textbf{algebraic set} or \textbf{affine variety}. The goal is thus to study the topology induced by the subspace topology of $\C^n$.

Up to this point, little ring theory is involved. However a non-negligible problem of this formulation is the loss of information in special cases where the space degenerate (in terms of dimension). For example, the following two spaces coincide for the underlying field being $\R$:
\[
    \{ (x, y) \in \R^2 \mid x = y = 0 \} 
    \quad \text{vs.} \quad
    \{ (x, y) \in \R^2 \mid x^2 + y^2 = 0 \}
\]
But the first set is merely a point while the second set is actually a dimension-1 manifold. Therefore, looking at the solution set itself may not be sufficient for understanding the topology of the underlying space.

A ``modern'' approach (1960s) is to encode the polynomials in an algebraic structure. Instead of looking directly at the set, consider the quotient ring $\operatorname{Spec}(\C[x_1, \dots, x_n]/(f_1, \dots, f_m))$ where Spec$(\cdot)$ is a functor to a \textbf{scheme}; and the whole object is referred to as an \textbf{affine scheme}. There are additional structures on the scheme that encodes the geometric properties (for example being ``tangential'').

\begin{remark}
    The extra layer of abstraction (having ring theory involved) has various advantages in terms of studying the structure of algebra:
    \begin{enumerate}
        \item Taking the perspective from the ring theory in a sense eliminates the problem of coincide under degeneracy. Elaborating on the example above, $x^2 + y^2 = 0$ and $x = y = 0$ coincide in $\R$, but the induced schemes $\operatorname{Spec} \R[x, y]/(x^2 + y^2)$ and $\operatorname{Spec} \R[x, y]/(x, y)$ are completely different schemes.
        \item It is hard to ``glue'' actual algebraic sets together, but gluing schemes is almost as easy as gluing topological spaces. An analogy that can be made here is the ``gluing'' of manifolds using charts.
        \item As we will see later, schemes as a topological space is not necessarily Hausdorff (every pair of points can be separated by a neighborhood). Specifically, points are necessarily closed. Although this seems to be weird, this indicates that there are actually fewer opens, i.e. the topology is much simpler. This turns out to be an advantage of the construction.
    \end{enumerate}
\end{remark}

\section{Presheaves and Sheaves}

\begin{definition}[Presheaf]
    A \textbf{presheaf} (of sets) $\mathcal{F}$ on a topological space $X$ is a pair $\{\mathcal{F}(\cdot), r\}$ where
    \begin{itemize}
        \item $\mathcal{F}(U)$ is a set for any $U$ open in $X$.
        \item A family of restriction maps $r_{uv}: \mathcal{F}(U) \to \mathcal{F}(V)$ for each $V \subseteq U \subseteq X$ opens, satisfying 
        \[
            r_{UU} = \Id_{\mathcal{F}(U)}, \qquad r_{VW} \circ r_{UV} = r_{UW}
        \]
    \end{itemize}
\end{definition}
\nogap
\begin{remark}
    A presheaf is equivalent to a contravariant functor.
\end{remark}

\begin{example}\label{ex: presheaf ex1}
    Let $Y$ be a topological space. A presheaf $\mathcal{F}$ can be defined by setting $\mathcal{F}(U) = \{ \text{continuous maps } U \to Y \}$ and $r_{uv}$ to be the restriction of maps.
\end{example}
\nogap
\begin{example}\label{ex: presheaf ex2}
    Consider the ``trivial'' presheaf. Let $S$ be a set. For all $U$ and $V$, take $\mathcal{F}(U) = S$, and $r_{UV} = \Id_S$.
\end{example}

\begin{notation}
    A presheaf $\mathcal{F}$ can be viewed as a structure defined ``above'' the topological space. One can expect that the notation will resemble that of vector bundles on manifolds (and there are more such analogies in subsequent objects):
    \begin{enumerate}
        \item Elements $s \in \mathcal{F}(U)$ are called \textbf{sections} of $\mathcal{F}$.
        \item $r_{UV}(s)$ for $s \in \mathcal{F}(U)$ is often denoted by $\restr{s}{V}$ (as ``restrictions'').
        \item $\mathcal{F}(U)$ will also appear using notation for sections in differential geometry, for example $\Gamma(U, \mathcal{F})$, or $H^0(U, \mathcal{F}) $.
    \end{enumerate} 
    When specifying a presheaf, one often omits the restriction maps if their definition is clear from context.
\end{notation}

\begin{definition}[Sheaf]
    A \textbf{sheaf} (of sets) $\mathcal{F}$ on a topological space $X$ is a presheaf that satisfies the following additional axioms:
    \begin{itemize}
        \item \emph{Identity axiom}: (``sections are determined by their restriction to small opens'') Let $U \subseteq X$ be an open, and $\{ U_i \}_{i \in I}$ is an open cover of $U$. Then for $s_1, s_2 \in \mathcal{F}(U)$
        \[
            s_1 = s_2
            \quad \Longleftrightarrow \quad
            \restr{s_1}{U_i} = \restr{s_2}{U_i} \quad \forall i \in I
        \]
        \item \emph{Gluability axiom}: (``compatible sections can be glued'') Let $U \subseteq X$ be an open, and $\{ U_i \}_{i \in I}$ is an open cover of $U$. Then for $s_i \in \mathcal{F}(U_i)$ and $s_j \in \mathcal{F}(U_j)$ s.t. $\restr{s_i}{U_i \cap U_j} = \restr{s_j}{U_i \cap U_j}$, there exists $s \in \mathcal{F}(U)$ s.t. $\restr{s}{U_i} = s_i$ and $\restr{s}{U_j} = s_j$.
    \end{itemize}
\end{definition}

\textstart
One can also combine the two axioms described above into one:

\begin{definition}[Sheaf, alternative]
    A \textbf{sheaf} on a topological space $X$ is a presheaf satisfying the \emph{sheaf axiom}: If $\{ U_i \}_{i \in I}$ is an open cover of $U$ and for all $i$, $s_i \in \mathcal{F}(U_i)$ agreeing on pairwise intersections. Then there exists a \emph{unique} $s \in \mathcal{F}(U)$ with $s_i = \restr{s}{U_i}$.
\end{definition}

\begin{remark}
    The gluability axiom specifies the existence of ``gluing'' of sections; and the identity axiom says that the resulting section is unique. It is clear that the two definitions are equivalent.
\end{remark}

\begin{example}
    It is then natural to examine whether the examples (Example \ref{ex: presheaf ex1}, \ref{ex: presheaf ex2}) we looked at in the introduction of presheaves are sheaves:
    \begin{enumerate}
        \item Recall the first example of presheaf, given by continuous maps $\mathcal{F}(U) = \{ U \tooh{\text{cont.}} Y \}$. By definition two functions are identical if and only if they agree on all points, and functions can be glued uniquely from common domains. Continuity can be checked from on opens in a cover, as the source functions are continuous (recall that the topological definition of continuity is that the preimage of opens are open).
        \item For the second example $\mathcal{F}(U) = S$, and $r_{UV} = \Id_S$ for all $U, V \subseteq S$ opens. This is actually \emph{not} a sheaf. Check the conditions respectively:
        \begin{itemize}
            \item \emph{Identity axiom}. This is not satisfied, as we have the special example $U = \emptyset$. Then $I = \emptyset \implies \{ U_i \} = \emptyset$. The axiom requires that any two elements in $\mathcal{F}(\emptyset)$ must be the same, but this will be not the case if the cardinality of $S$ is greater than 1. (That is, the restriction of opens to empty sets are not unique.)
            \item \emph{Gluability axiom}. This is actually satisfied, as we can take apply the restriction map to the section $\mathcal{F}(U) = S$. Restriction maps here are uniquely determined as identities.
        \end{itemize}
    \end{enumerate}

    Now we try to fix the second example to make it a sheaf: 
    \begin{itemize}
        \item An intuitive fix would be to fix $\mathcal{F}(\emptyset) = \{ \cdot \}$, a singleton set. This corresponds to $\mathcal{F}(U) = \{ U \tooh{\text{const.}} S \}$. But this fails gluability condition: Let $\mathcal{F}(U) = \Z$, and $U_1, U_2 \subseteq U$ disjoint with $\mathcal{F}(U_1) = 3, \mathcal{F}(U_2) = 4$. Transitivity of restrictions require $\mathcal{F}(\emptyset)$ contains the intersection of $\{3\}$ and $\{4\}$, which contradicts the premise that $\mathcal{F}(\emptyset)$ is a singleton.
        \item This, however, can be fixed by restricting the constant condition to local, e.g. taking
        \begin{align*}
            \mathcal{F}(U)
            & = \{ f: U \to S \text{ \underline{locally constant}, i.e. } \forall p \in U, \exists \text{ neighborhood where $f$ is constant} \} \\
            & = \{ U \tooh{\text{cont.}} S, \text{where $S$ is given the \underline{discrete topology}, i.e. all subsets are open} \}
        \end{align*}
    \end{itemize}
    \ 
\end{example}
\nogap
\begin{remark}
    The intuition of the (counter-)example above is that the sheaf condition can only make local restrictions (similar to charts in manifolds), as difference in global conditions cannot coincide on small opens. For example, $\mathcal{F}(U) = \{ U \to \R \text{ bounded} \}$ is not a sheaf on $X = \R^n$ as ``gluing'' sheaves would require taking infinitely many opens with different limits, thus resulting in a sheaf that is unbounded. A fix would be change the condition to $\mathcal{F}(U) = \{ U \to \R \text{ locally bounded} \}$.
\end{remark}

\begin{definition}[Constant Sheaf]
    The resulting sheaf from the fix above is the \textbf{constant sheaf}. Given a topological space $X$ and a set $S$, the sheaf is denoted $\underline{S}$, i.e.
    \[
        \underline{S}(U) := \{ f: U \to S \text{ locally constant} \}
    \]
\end{definition}

\section{Operation on Sheaves}

\textstart
The next step is to construct sheaves based on existing ones. The idea would be viewing a sheaf on topological space $X$ as an additional structure on $X$ controlled by small opens, specified by the topology, i.e. thinking of $(X, \mathcal{F})$ as some ``augmented'' topological space.

A first trivial example will be simply restricting the sheaf to a open subsets: if $\mathcal{F}$ is a sheaf on $X$ and $V \subseteq X$ open, then there is a sheaf $\restr{\mathcal{F}}{V}$ defined by $\restr{\mathcal{F}}{V}(U) = \mathcal{F}(U)$ for all $U \subseteq V$ open (and also $U$ open in $X$).

Using the similar analogy to vector bundles, sheaves can be transitioned along maps:

\begin{definition}[Pushforward Sheaf]\label{def: pushforward sheaf}
    If $\mathcal{F}$ is a shaef on $X$, and $\pi: X \to Y$ continuous, then there is a sheaf $\pi_{\ast}\mathcal{F}$ on $Y$ defined by $\pi_{\ast}(\mathcal{F})(U) = \mathcal{F}(\pi^{-1}(U))$ for all $U \subseteq Y$ open.
\end{definition}
\nogap
\begin{remark}
    The pushforward sheaf is indeed a sheaf, as the open covers are transition by continuous maps. If $\{ U_i \}_{i \in I}$ is an open cover of $U \subseteq Y$ open, then $\{ \pi^{-1}(U_i) \}_{i \in I}$ is an open cover of $\pi^{-1}(U) \subseteq X$.
\end{remark}

For example, take $X = \{ \cdot \}$ a singleton set, and $Y$ any topological space. Let $p \in Y$ be a point, and $S$ be an arbitrary set. Take the constant sheaf $\underline{S}$ on $X$, and the continuous map $\iota_p: X \to Y, \cdot \mapsto p$. This pushforward sheaf is special and we give it a definition:

\begin{definition}[Skyscraper sheaf]
    The \textbf{skyscraper sheaf} is the sheaf $(\iota_p)_{\ast} \underline{S}$, i.e. explicitly
    \[
        (\iota_p)_{\ast} \underline{S}(U) = \underline{S}(\iota_p^{-1}(U)) =
        \begin{cases}
            S & \text{if $p \in U$} \\
            \{ \cdot \} & \text{if $p \notin U$}
        \end{cases}
    \]
\end{definition}

\textstart
The name of the skyscraper sheaf can be interpreted as the sheaf being ``concentrated'' at the point $p$, and sort of trivial elsewhere.

\begin{example}
    For understanding this example first supplement some necessary definitions:

    \begin{definition}[Covering]
        Let $X$ be a topological space. A \textbf{covering} of $X$ is a continuous map $\pi: \widetilde{X} \to X$ s.t. for every $x \in X$, there exists a neighborhood of $U_x$ of $x$, and index space $I_x$ s.t. $\pi^{-1} (U_x) = \bigsqcup_{d \in I_x} V_d$ and $\restr{\pi}{V_d}: V_d \to U_x$ is a homeomorphism for all $d \in I_x$.
    \end{definition}

    Let $\pi: \widetilde{X} \to X$ be a covering. Then $\pi_{\ast} \underline{S}$ encodes the number of connected components in $\pi^{-1}(U)$ for all $U$ open in $X$, as local constant properties extend to opens, which form connected components.
\end{example}

\begin{definition}[Sheaf of Section]
    Let $\pi: Y \to X$ be a continuous map. Then the \textbf{sheaf of section} of $\pi$ is a sheaf $\mathcal{F}$ on $X$ given by
    \[
        \mathcal{F}(U) = \{ s: U \to Y  \text{ continuous sections of $\pi$ above $U$}\}
    \]
    i.e. making the following diagram commutative:

    \begin{minipage}{\linewidth}
        \centering
        \begin{tikzcd}
            Y \arrow[rrdd, "\pi"] & & \\
            & & \\
            U \arrow[uu, "s"] \arrow[rr, hookrightarrow] & & X
        \end{tikzcd}
    \end{minipage}
    Restriction of functions are given by restriction of sections.
\end{definition}
\nogap
\begin{example}
    Consider the projection $\pi: Y \times X \to X$. Then $\mathcal{F}(U)$ are simply continuous maps from $U$ to $Y$.
\end{example}

\begin{remark}
    Every sheaf is isomorphic to the sheaf of sections of some continuous map, analogous to the construction above. For example, for the skyscraper sheaf, $(\iota_p)_{\ast} S$ is isomorphic to the sheaf of sections where $Y = X \smallsetminus \{ p \} \cup S$, with topology properly defined s.t. the map $Y \to X, s \mapsto p$ and identity elsewhere is continuous.
\end{remark}

\section{Categorical Formulation of Sheaves}

\textstart
Previously our definition of sheaves is restricted to sets. However, we have not used any additional property other than its elements. This then can be generalized to other categories. 

Since the objects are clear, we only need to specify the morphisms between sheaves to have a category. An intuitive idea may be defining morphisms between arbitrary sheaves defined on arbitrary topological spaces. However the morphisms between the topological spaces may not be straightforward, and instead we require that all objects in the category should be defined on the same topological space. (Although in the upcoming definition on \underline{scheme}s, we will define morphisms like above).

\begin{definition}[Category of (Pre)Sheaves on Sets]
    Let $X$ be a topological space. The \textbf{category of sheaves} (of sets) on $X$, denoted $\catsets_X$, is defined by providing the following data:
    \begin{itemize}
        \item The objects of $\catsets_X$ are given by sheaves on $X$.
        \item For $\mathcal{F}, \mathcal{G} \in \Ob(\catsets_X)$, a morphism $\pi: \mathcal{F} \to \mathcal{G}$ is a function $\pi(U): \mathcal{F}(U) \to \mathcal{G}(U)$ for each $U \subseteq X$ open s.t. it commutes with restrictions, i.e. the following diagram commutes:
        
        \begin{minipage}{\linewidth}
            \centering
            \begin{tikzcd}
                \mathcal{F}(U) \arrow[rr] \arrow[dd, "\restr{(\cdot)}{V}"] & & \mathcal{G}(U) \arrow[dd, "\restr{(\cdot)}{V}"] \\
                & & \\
                \mathcal{F}(V) \arrow[rr] & & \mathcal{G}(V)
            \end{tikzcd}
        \end{minipage}
        \item Identity morphism and composition of morphisms are given by the identity function and composition of functions.
    \end{itemize}
    Notice that when defining morphism we did not use any portion of the sheaf axiom. Therefore, we can define the category of presheaves on $X$ $\catsets_X^{\text{pre}}$ by replacing all occurrences of ``sheaf'' with ``presheaf''.
\end{definition}

\begin{example}
    Let $\mathcal{F}$ be the sheaf on $X$ of continuous functions to $R$, i.e. $\mathcal{F}(U) = \{ U \tooh{\text{cont.}} \R \}$. Then $\pi: \mathcal{F} \to \mathcal{F}, s \mapsto 2s$ (in the sense of functions) is a morphism of sheaves. $\pi$ commutes with restrictions as scalar multiplication commutes with restriction for continuous functions.

    Let $\mathcal{G} = (\iota_p)_{\ast} \underline{\R}$ on $X$, i.e. the skyscraper sheaf with set $\R$ at $p \in X$. Consider the morphism $ev: \mathcal{F} \to \mathcal{G}$ defined as
    \[
        \mathcal{F}(U) \ni s \mapsto
        \begin{cases}
            s(p) \in \R = \mathcal{G}(U), & \text{if $p \in U$} \\
            \cdot \in \ \{ \cdot \} = \mathcal{G}(U), & \text{if $p \notin U$}
        \end{cases}
    \]
    $ev$ also commutes with restrictions, as the evaluation will only change when the restriction to $V$ loses $p$; which is considered in both the morphism and restriction in $\mathcal{G}$.
\end{example}

\begin{remark}
    In the previous section, we have introduced the restriction of sheaves and \hyperref[def: pushforward sheaf]{pushforward sheaves} $\mathcal{F} \mapsto \restr{\mathcal{F}}{V}$ and $\mathcal{F} \mapsto \pi_{\ast}\mathcal{F}$. They can be viewed as functors between the categories:
    \[
        \restr{(\cdot)}{V}: \catsets_X \to \catsets_V, \quad \pi_{\ast}: \catsets_X \to \catsets_Y
    \]
    It is clear from the definition how they transit objects. For morphisms, \TODO{finish}
    
\end{remark}

\end{document}
